%% LyX 2.3.7 created this file.  For more info, see http://www.lyx.org/.
%% Do not edit unless you really know what you are doing.
\documentclass[english]{article}
\usepackage[T1]{fontenc}
\usepackage[latin9]{inputenc}
\usepackage{bm}
\usepackage{amssymb}
\usepackage{babel}
\begin{document}

\section{Do auto-regressive models bite their own tail? The value of Gedankenexperiments}

Autoregressive models use their output to arrive at predictions. In
machine learning, this amounts to training on the output, i.e. generated
data. Their existence justifies this apparent conundrum, there seems
to be a benefit. 

Or is there? The data processing inequality asserts that no pipeline,
however deep, can arrive at new information. How can this be reconciled?

\subsection{The autoregressor}

Consider a stream of data samples, $\bm{x}_{\leq t}=\left(x_{1},\ldots,x_{t}\right)$,
for example elements of a time series, or images of cats. The autoregressor
then is a model $p$ that generates the next output, 
\[
\hat{x}_{t+1}\sim p\left(\bm{x}_{\leq t}\right),
\]

where we denote model-generated outputs with a hat $\hat{x}$. The
model now becomes autoregressive if part of its input is its output. 

\subsection{The data-processing inequality}

The data-processing inequality posits that for any processing pipeline
$Y=f(X)$, the pipeline cannot increase the information that the input
contains about the source, 
\[
\mathbb{I}\left[Z=f(Y);X\right]\leq\mathbb{I}\left[Y;X\right]\quad\forall\,f.
\]

Now, the autoregressor suggests that such a pipeline does exist: Namely,
let's consider a ground truth distribution that we want to model,
$p^{*}$. Ideally, the model should be as close as possible to $p^{*}$,
that is the loss $\mathbb{I}\left[\hat{p};p^{*}\right]$ should be
minimized. However, there is only a finite amount of samples from
$p^{*}$ available to fit $\hat{p}.$ Can we hence use autoregressive
feedback to define a sequence of predictors
\[
X\rightarrow\hat{p}^{(1)}\sim X^{(1)}\rightarrow\hat{p}^{(2)}\sim X^{(2)}\rightarrow\ldots\rightarrow\hat{p}^{(n)}\sim X^{(n)}
\]

such that $\mathbb{I}\left[X;X^{(n)}\right]\geq\mathbb{I}\left[X;X^{(1)}\right]$?

\subsection{Real World}

Imagination, hallucination or dreaming are basically autoregressive:
Decoupled from a finite data sample, the brain ponders what it remembers,
hopefully arriving at a better model. Even more clearly, closing your
eyes and giving things a good thought can definitely improve your
actions. 

Another expression of this is {[}prove{]}

\[
\hat{p}(y|x,YX)=\langle\hat{p}(y|x,\hat{Y}\hat{X},YX)\rangle_{\hat{Y},\hat{X}\sim\hat{p}(y|x,YX)}.
\]

I.e., sampling from your existing model on average does not improve
the model. 

So what is it good for? 

\subsection{Reasoning}

So far, we have adopted a puristic Bayesian inference view. However,
in practice information needs not only to be present, but \emph{be
made accesible}. This helps to reconcile it with our everyday intuition
of pondering indeed being helpful for action. 

The idea is that sampling from even a premature model will help with
exploration: Maybe you'll get an idea that helps you downstream. 
\end{document}
